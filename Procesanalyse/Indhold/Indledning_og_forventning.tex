Denne analyse har til formål at dokumentere P1 projektforløbet for gruppe B128. Derudover er formålet at reflektere over samarbejdet og arbejdsgangen i gruppen.
\\
Gruppen bestod af de samme personer som i P0, og i P0-projektforløbet, gik samarbejdet godt og vi fungerede fint sammen socialt. Derfor blev det besluttet at fortsætte i P1 i samme gruppe og vi valgte så et emne, som vi alle synes var spændende. Derved kendte vi alle hinanden og det gjorde at det var let at komme i gang med projektet, da vi ikke skulle bruge tid på at lære hinanden at kende. Vi havde i gruppen en forventning om at projektet skulle forløbe nogenlunde som det havde gjort i P0. Derudover havde alle en forventning om at ambitionsniveauet skulle være højt og projektet skulle ende med at ligge over middel. 
\\
Med hensyn til udbyttet af projektet, havde vi en forventning om, at lærer mere om at skrive en god rapport og at for gruppearbejdet til at køre. Derudover havde vi også bestemt os for at alle i gruppen skulle lære at bruge \LaTeX{} og ligeledes skulle vi lære at bruge GitHub i stedet for Dropbox, som blev brugt i P0.  
