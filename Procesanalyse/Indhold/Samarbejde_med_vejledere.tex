\section{Beskrivelse}
Vi havde vores første vejledermøde den 16/10, her var både hovedvejleder og bivejleder tilstede. Inden mødet blev vi bedt om at tænke over hvad vi forstod ved opgaveformuleringen, samt overveje hvad vi ville bruge vejlederne til. 
Som svar på første ”spørgsmål” kom vi frem til følgende:
\begin{itemize}
\item	Definere en model, der kan komprimere og dekomprimere

\item Udvikle et program, som virker til en valgt platform

\item	App/program?
\end{itemize}
Med hensyn til hvad vi ville bruge vejlederne til, kom vi med følge punkter:
\begin{itemize}
\item	Vi vil gerne løbende sende hvad vi laver, pr.mail, til henholdsvis vejleder og bivejleder, afhængig af hvad det drejer sig om. Og så håber vi selvfølgelig på at få feedback.

\item	 Vi vil gerne holde møder, når der er brug for det, med henblik på at tale om hvordan det går i projektet og få afklaret eventuelle spørgsmål.

\item	Derudover vil vi også gerne have mulighed for løbende at sende spørgsmål over mail, når de skulle melde sig.
\end{itemize}Begge vejleder mente at det var en fornuftig måde at bruge dem og vi blev enige om at samarbejdet skulle fungere på den måde. Det har kørt sådan, at vi har sendt en mail til den vejleder vi gerne ville have møde med, når vi følte at vi havde bruge for dette. I mailen blev vores foreløbige rapport vedhæftet og til mødet talte vi så om hvad vi burde rette og hvad status på projektet var. Vi har ikke fået brugt vores bivejleder så meget. I perioder var vi meget dårlige til at underrette vejlederne om status på projektet og fik ikke holdt nok møder. Det gjorde at vejlerne var nødt til selv at tage kontakt til gruppen, for at høre hvordan det gik og om det ikke var tid til at få holdt et møde. Ligesom med arbejdsindsatsen, er der blevet holdt flest møder og været mest kommunikation med vejlederne op til statusseminar og projektaflevering. 
Til hvert vejleder møde, blev der valgt en referent, så vi altid fik skrevet referat. Til gengæld fik vi ikke udarbejdet dagsorden inden hvert mødte, selvom det egentlig var vores plan fra starten. 

\section{Analyse}

\emph{Gode erfaringer:}
\begin{itemize}
\item	Godt at tale med vejlederne om hvad vi forventer af dem.

\item Godt at maile mødetidspunkter, det var en nem måde at kommunikere på og på samme tid kunne vi sende foreløbige rapport.

\item At vælge en referent inden mødet var godt, så var vi sikre på at vi ikke glemte noget.
\end{itemize}\emph{Dårlige erfaringer:}
\begin{itemize}
\item	Vi kunne godt have gjort bedre brug af bivejleder især i starten af projektet. Så kunne vi måske hurtigere have afsluttet analyse-delen, så vi ikke hele tiden vendte tilbage til den, som var det der skete.

\item	Med en dagsorden inden hvert møde, kunne vi være mere sikre på, at vi huskede at få talt med vejlederen, om alt det vi gerne ville.

\item	Ved at vælge én i gruppen, til at være kontaktperson til vejlederne, kunne man være mere sikre på at rapport blev sendt rettidigt til vejlederne. Og at der blev spurgt om møde når der var behov for det.

\item	Når der gik for langt imellem at vi havde kontakt med vejlederne, fik vi lavet for lidt og risikerede også at arbejde ud af et forkert spor.
\end{itemize}

\section{Forbedringer til P2}

I næste projekt skal vi have bedre kommunikation med vejlederne. Hvilket vil sige, at vi skal holde dem opdateret med hvad vi laver løbende, og ikke bare inden statusseminar og aflevering. For at være sikre på at gøre dette, kunne man fra starten tale med vejleder om hvor ofte vi de gerne vil opdateres og derefter kunne man skrive ind i gruppekontrakten, hvem der har ansvaret for at gøre dette. 
I næste projekt skal gruppen forberede sig mere seriøst inden et vejledermøde, ved eksempelvis at udarbejde en dagsorden og gennemarbejde de spørgsmål der var til vejlederen. På den måde kunne man få et større udbytte af hvert møde, og både gruppe og vejleder ville sikkert føle at alt det nødvendige var blevet diskuteret. 
