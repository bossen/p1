\section{Beskrivelse}
I starten af projektet skrev vi en gruppekontrakt. Vi skrev mest for at prøve det og øve os i at udarbejde en skriftlig aftale. Vi synes den var lidt svær at lave, fordi gruppen i P0 projektet ikke havde haft konflikter eller andre problemer, som man kunne have beskrevet hvordan skulle håndteres, i en kontrakt. Som sagt havde alle i gruppen en forventning om at projektet skulle køre som det havde kørt i P0. Derfor blev kontrakten ikke særlig stram og handler nok mest om en kageordning. Vi blev dog i kontrakten enige om, at der var ”mødepligt” til både forelæsninger, vejledermøder og gruppemøder. Hvilket vil sige, at hvis et gruppemedlem af en given årsag ikke kunne møde op til én af de tre nævnte møder, skulle der meldes afbud. Denne regel blev dog hurtigt brudt af gruppens 7. medlem, som begyndte at blive væk fra forelæsninger og gruppemøder uden at give besked til resten af gruppen. Som tiden gik, blev dette værre og værre og til sidst fik vi i gruppen heller ikke noget svar på nogen af de beskeder vi sendte til det 7. medlem. Efter at have snakket med studiesekretæren og vores vejleder, blev det bestemt at vi måtte gå ud fra at vores 7. mand var stoppet på uddannelsen. Derved blev vi en 6 mandsgruppe og fik lige pludselig lidt mere at lave en vi havde regnet med.
I gruppen var der ikke nogen faste mødetidspunkter, men det blev aftalt fra dag til dag hvornår vi mødtes i grupperummet næste dag. Indimellem aftalte vi hvad alle skulle lave efterfølgende dag og hvis der så ikke var nogen forelæsninger på skemaet, sad vi hjemme og lavede vores egen del. 
Vi har brugt Facebook rigtig meget til kommunikation med hinanden, når vi sad hjemme og ikke var samlet i grupperummet. Vi har på Facebook.com, oprettet en gruppe, som er en side kun for vores gruppe. På denne side, har vi aftalt mødetidspunkter og delt forskellige hjemmesider, som har været interessante for hele gruppen, at læse. Når vi så har været samlet i grupperummet, har vi brugt tavlerne meget og det har været rigtig godt til at skabe overblik. Vi har eksempelvis skrevet spørgsmål til vejleder op på tavlen, for ikke at glemme dem og en to-do liste over dagen. Mod slutningen af projektet skrev vi også på en tavle hvad der konkret manglede at blive gjort før at projektet var færdigt, og disse blev så streget ud i takt med at de blev udført. På den måde kunne alle følge med i hvad der blev lavet og hvad der stadig manglede. 

I gruppen har der været enighed om at lave ting sammen uden for grupperummet. Dette var med henblik på at styrke sammenholdet og samarbejdet i gruppen. Vi har eksempelvis haft ”spille-aften”, spist aftensmad sammen, gået i fredagsbar og i byen sammen. Derved har vi opnået at alt ikke bare drukner i fagligt arbejde, men at vi også kan have det sjovt sammen.

Kommunikationen i gruppen har været godt og ret ligeligt fordelt. Ingen i gruppen har opfattelse af, at én i gruppen taler mere end de andre. På samme tid er der heller ikke opfattelse af at én i gruppen ikke bliver hørt. Derimod mener hele gruppen, at alle byder ind, når de har noget relevant at sige og at alle bliver hørt. På den måde er der plads til alle i gruppen og derved har samarbejdet fungeret optimalt gennem hele projektforløbet.

\section{Analyse}

\emph{Gode erfaringer:}
\begin{itemize}
\item	Godt med kageordning hver fredag, det gav motivering og glæde på ugens sidste dag. 

\item	”Mødepligt” til forelæsninger motiverede gruppen endnu mere til at komme hver gang. 

\item	Fleksible mødetider var godt, fordi man ikke følte sig låst fast og nogen i gruppen, synes også bedre om at arbejde hjemme.  

\item	Godt med Facebook til intern kommunikation, det er en side man alligevel er inde på hver dag.

\item	Sociale arrangementer udenfor arbejdstiden, var rigtig godt til at lærer hinanden at kende på en anden måde og gjorde at når vi skulle arbejde, var vi seriøse. 

\item	 Brug af tavler i grupperummet var godt til at skabe overblik og holde gruppen fast på hvad der skulle laves. 
\end{itemize}\emph{Dårlige erfaringer:}
\begin{itemize}
\item	Gruppekontrakten bør være strammere og mere udførlig, ellers får man intet ud af den.

\item	 At én i gruppen ”melder sig ud”, uden at meddele det til resten af gruppen giver en masse frustration og tidsspild.

\item	Mødetiderne blev ikke altid overholdt, hvilket gav frustration til den del af gruppen der var mødt til tiden.
\end{itemize}

\section{Forbedringer til P2}

I efterfølgende projekter ville det være godt at bearbejde gruppekontrakten mere. Der bør være en konsekvens hvis man kommer for sent, så man på den måde motivere hele gruppen til at møde til tiden. Derudover 
