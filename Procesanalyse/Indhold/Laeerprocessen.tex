\section{Beskrivelse}
Sideløbende med projektarbejdet, har vi deltaget i 3 kurser; Lineær Algebra, PV og Imperativ Programmering (IMPR). Til brug i projektet har vi fået mest ud af IMPR og PV. I PV har vi lært meget om hvordan den gode analyse ser ud, hvordan den gode problemformulering er bygget op og hvordan den gode rapport er struktureret.
I PV blev der introduceret forskellige læringsstile-tests, som hele gruppen tog. Derved fandt vi ud af på hvilken måde vi bedst kan tilegne os nyt stof. Gruppen var rimelig ens hvad dette angår og vi fandt ud af at alle generelt bedst kan forstå visuelle ting, som grafer, billeder osv.. Denne viden har vi gjort brug af i projektet, ved at vi har brugt tavlerne i grupperummet til at tegne på, når vi skulle forklare de andre noget nyt og vi har også brugt mange billeder og figurer i selve rapporten. 
I IMPR lærte vi naturligvis det nødvendige for at kunne udarbejde et stykke software, som der er krav om i dette projekt.

\section{Analyse}

\emph{Gode erfaringer:}
\begin{itemize}
\item	Læringsstile-testene gjorde at vi vidste hvordan gruppen havde nemmest ved at lære nyt stof. 

\item	 Når et afsnit til rapporten var blevet færdiggjort, skulle mindst to andre læse det igennem. Når dette blev overholdt, gjorde det at man hele tiden lærte om hele projektet og ikke bare den del man selv havde skrevet. 

\item	Godt at der altid var nogen til alle forelæsninger, så gruppen ikke gik glip af noget.
\end{itemize}\emph{Dårlige erfaringer:}
\begin{itemize}
\item	Hvis hele gruppen var mødt op til alle kursusgangene i PV og havde arbejdet på de opgaver vi her fik stillet, kunne det have hjulpet til nemmere at få skrevet et godt og struktureret projekt. 

\end{itemize}	 

\section{Forbedringer til P2}
I næste projekt skal vi have "mødepligt" til alle kurser og ikke vælge at ét kursus er mindre vigtigt og man derfor ikke behøver møde op. Vi skal også gøre mere for at alle løbende forstår hele projektet og alle derved lærer det samme. På den måde kan vi måske undgå at gruppemedlemmerne ved mest om den del af projektet de selv har skrevet. 