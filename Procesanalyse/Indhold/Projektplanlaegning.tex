\section{Beskrivelse}
Vi har tidligt i projektet udarbejdet en tidsplan over hele projektperioden, hvori vi lavede deadlines for hvornår de forskellige ting i projektet skulle være færdiggjort. Vi har dog ikke været særlig gode til at benytte os af planen og vi har heller ikke holdt den opdateret. I stedet har vi delt opgaver ud fra dag til dag og aftalt tidspunkter for hvornår opgaverne skulle være færdiggjort. Ved vores daglige arbejde i grupperummet opgjorde vi status af projektet og aftalte ting som hvornår vi skulle mødes næste dag. 

I P0-projektet, brugte vi Dropbox til fildeling. Det blev lidt besværligt, at hvis 2 personer havde den samme fil åben på engang, blev der skabt en conflicted copy. Derfor valgt vi denne gang at bruge GitHub til at dele filer mellem os og det virkede rigtig godt. Vi brugte dog noget tid i starten af projektet på at få det til at fungere, men det var tid der var godt givet ud. 

Vi valgte fra starten af projektet, at vi ikke skulle have en projektleder. Der var altså ikke én person der havde styringen, derimod havde alle ansvar for at holde arbejdet i gang og vi skiftedes til at være motivator, når der var brug for det. Der blev heller ikke delt andre grupperoller ud.  

Midt i projektforløbet, blev der afholdt statusseminar. Her fik vi gode idéer og forslag til rettelser fra både vejledere og opponentgruppe. Til seminaret valgte vi at 4 ud af gruppens 7 medlemmer skulle fremlægge, hvorefter alle 7 selvfølgelig deltog i diskussionen. Efter statusseminaret rettede vi selvfølgelig det, som vi havde fundet ud af ikke var godt nok, og derefter havde vi faktisk lidt svært ved at komme videre. På GitHub.com kan man se en graf over hvor meget aktivitet der har været i projektet i forhold til tiden, se bilag \ref{bilag2}. Her fremgår det tydeligt at, vi kom lidt langsomt fra start og lavet rigtig meget kort forinden statusseminaret. Efter seminaret stod det igen lidt stille og først hen mod aflevering af projektet har vi lavet meget igen. 

\section{Analyse}

\emph{Gode erfaringer:}
\begin {itemize}
\item  GitHub var et rigtig godt alternativ til Dropbox – det vil vi klart benytte i fremtidige projekter.

\item	At aftale mødetidspunkt for dag til dag – når folk ellers mødte til tiden. 

\item	Statusseminaret var rigtig godt til at øve en fremlæggelse af det foreløbige projekt og til at få nye ideer.

\item	LaTex er et rigtig godt program til at skrive rapport i.
\end{itemize}\emph{Dårlige erfaringer:}
\begin{itemize}
\item	Vi arbejdede først igennem lige optil en deadline (statusseminar eller endelig aflevering). Det havde været bedre med en konstant arbejdsindsats. 

\item	Ikke at have en ordentlig tidsplan, som blev opdateret, gjorde at vi indimellem mistede overblikket over projeket. 

\item	At tilføje noget til rapporten uden det er ordentlig læst igennem, af mindst to personer.

\item	Man kunne vælge én person, som skulle tage sig af praktiske ting, som kontakt med vejleder, opdatere tidsplan osv. Man kunne skiftes til at have rollen hver uge eksempelvis. 
\end{itemize}

\section{Forbedringer til P2}

I næste projekt skal vi forbedre brugen af en struktureret plan over projektet, så vi undgår at miste overblikket, som det ind imellem er sket i denne projektperiode. Derudover skal vi fra starten uddele ansvaret for de forskellige dele af projektet mellem alle mand, så vi er sikre på at alt bliver lavet. Der skal også være mere fokus på tidsplanen gennem projektet, så der kommer flere deldeadlines. På den måde kan vi måske undgå at arbejdsindsatsen bliver så ”bølget”, og vi kan undgå at der kommer som en overraskelse at der eksempelvis kun er 2 uger til afleveringer.

Det ville nok også være en god ide at have en eller anden form for rollefordeling. En projektleder, sekretær og lignende. Man kunne bytte hver uge for at alle kunne prøve at have de forskellige roller. Måske kunne projektet på den måde blive mere struktureret.
