Hvis projektet skulle laves forfra, så skulle der forsøges at indsamle data om interessenterne for løsningen. Det ville eksempelvis være interessant at vide hvor mange SMS-beskeder de sender, både indenrigs og udenrigs. Derudover ville det selvfølgelig være vigtigt at vide hvor mange af deres SMS-beskeder, der overstiger 160 tegn. Til at indsamle denne kvantitative data, ville en spørgeskemaundersøgelse være en god mulighed. Ved at have denne data, kunne man regne på den faktiske besparelse på SMS-forbrug for et firma eller privatpersoner, hvis de implementerede dette projekts løsning. Herved kunne der blive vurderet, om programmet ville have en effekt af betydning. 

Derudover kunne det også være relevant, at kigge på om løsningen ville påvirke teleselskaberne, og hvorvidt det ville være positivt eller negativt. Spørgsmålet er om teleselskaberne ville miste penge ved at deres kunder sender færre SMS-beskeder, eller om de selv kunne spare penge, hvis løsningen blev implementeret i deres netværk. Derudover kunne man også undersøge i hvilke andre sammenhænge man ellers med fordel kan bruge datakomprimering. For eksempel når man skal lagre filer og ved generel internettrafik. Ud fra det kunne man finde ud af om løsningen kunne være relevant i andre sammenhænge.

Hvis programmet skulle bruges i al almindelighed ville det kræve at alle mobiltelefoner skulle have programmet installeret. Ellers vil en modtager, som ikke har programmet, læse den komprimerede besked, og denne giver ikke megen mening. Dette er nødvendigt fordi løsningen gør brug af et statisk træ, som er et generelt træ for hvor tit tegn forekommer i en tekst. Det vil sige at alle mobiltelefoner skulle opdateres til at indeholde programmet. Hvis man gjorde brug af et dynamisk eller adaptivt træ, så er det stadig en nødvendighed at modtageren har et eller andet installeret, sådan at enheden kan dekomprimere og læse beskeden. Dette er et krav, uanset hvilken form for komprimeringsalgoritme man gør brug af. En måde hvorpå man kan lettere komme omkring denne forhindring, kunne være et fremtidigt mål for programmet.
 
Derudover er det forventet, at komprimerings programmet vil  blive mest udbredt blandt smartphones, idet de har noget lettere tilgang til ny software eller applikationer. Dermed er det ikke sagt, at dette projekts løsning ville være umulig at installere på en mobiltelefon, det ville bare være lidt mere krævende end at downloade en applikation til en smartphone. Et fremtidigt mål kunne derfor være at finde en måde, sådan at det er let at implementere løsningen på både almindelige mobiltelefoner og smartphones. 

Til det nuværende statiske Huffman træ, som bliver brugt til at komprimere SMS beskeder, blev der indsamlet omkring 14.000 SMS beskeder. En mængde på 14.000 SMS beskeder giver et godt træ, som er forholdsvist præcis og tilpasset til skrivestilen ved SMS beskeder. Træet bliver dog kun bedre hvis mængden af SMS bliver større, idet træet bliver mere præcist, og tilpasser sig mere skrivestilen for den generelle bruger, end en håndfuld af personer. Et mål til at forbedre træet kunne derfor være at indsamle langt flere SMS beskeder, samt finde en måde at indsamle SMS beskeder som tager højde for brugerens privatliv.
  
I fremtiden kunne man muligvis overveje, hvorvidt PPM kan blive implementeret. PPM kan blive bygget ovenpå den allerede eksisterende Huffman komprimeringsalgoritme, og derved gøre løsningen langt bedre. Derudover skal der også laves nogle flere overvejelser, om hvor meget regnekraft og plads man har tilgængelige på en mobiltelefon eller smartphone. Det er muligt, at især smartphones har nået et punkt hvor de sagtens kan håndtere tungere programmer, som derfor åbner op for mulige løsninger ved at bruge PPM, og derudover også bruge dynamiske eller adaptive Huffman træer.