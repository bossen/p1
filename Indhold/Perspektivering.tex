Hvis vores program skulle bruges i al almindelighed ville det kræve at alle mobiltelefoner skulle have programmet installeret. Ellers vil en modtager, som ikke har programmet, læse den komprimerede besked, og denne giver ikke megen mening. Dette er nødvendigt fordi løsningen gør brug af et statisk træ, som er et generelt træ for hvor tit tegn forekommer i en tekst. Det vil sige at alle mobiltelefoner skulle opdateres til at indeholde programmet. Hvis man gjorde brug af et dynamisk eller adaptivt træ, så er det stadig en nødvendighed at modtageren har et eller andet installeret sådan at enheden kan dekomprimere og læse beskeden.
 
Derudover forventer vi, at vores løsning ville blive mest udbredt blandt smartphones, idet de har noget lettere tilgang til ny software eller applikationer. Dermed er det ikke sagt at vores løsning ville være umulig at installere på en mobiltelefon, det ville bare være lidt mere krævende end at downloade en applikation til en smartphone. 
  
I fremtiden kunne man muligvis overveje, hvorvidt PPM kan blive implementeret. PPM kan blive bygget ovenpå den allerede eksisterende Huffman komprimeringsalgoritme, og derved gøre løsningen langt bedre. Derudover skal der også laves nogle flere overvejelser, om hvor meget regnekraft og plads man har tilgængelige på en mobiltelefon eller smartphone. Det er muligt, at især smartphones har nået et punkt hvor de sagtens kan håndtere tungere programmer, som derfor åbner for mulige løsninger ved at bruge PPM, og derudover også bruge dynamiske eller adaptive Huffman træer.