I dette afsnit ses på hvordan programmet er blevet testet, i forhold til de opstilte krav og 
forudsætninger. Der vil blive forklaret hvilken fremgangsmåde der blev benyttet, samt visualiseringer af 
resultaterne.

For at måle den gennemsnitlige komprimeringsgrad programmet kunne opnå, benyttede vi en række samlinger 
af SMS’er, fra forskellige personers mobiltelefoner med forskellige SMS’ præferencer. Hermed menes at der 
var stor forskel på længden af beskederne, men også det skriftlige sprog, heriblandt brugen af 
humørikoner.
Disse samlinger af beskeder blev samlet i en enkel fil, som blev komprimeret med programmet.
Denne fremgangsmåde blev gjort med både GSM bit-træet, skabt ud fra SMS’er, og med bit-træet skabt med 
artikler fra Wikipedia. Dette gør det muligt at sammenligne effektiviteten af det statiske bit træ, som 
er lavet til SMS’ers formål, hvorimod træet ud fra Wikipedia, henvender sig bedre til artikler og et 
andet ordforråd end det typisk benyttede i SMS’er.
Herved er det muligt at sammenligne størrelsen på den komprimerede fil, i forhold til den originale fil 
med de læsbare SMS’er.

På figur /ref{resultatgraf} er forskellen på en gennemsnitlig komprimeret SMS sammenlignet med en 
gennemsnitlig original SMS.

På grafen ses tre forskellige typer dele som udgør blokken. Filstørrelsen er den største del, som ved den 
originale besked udgør hele filen.

Anden del, er de tegn som ikke er en del af GSM standarden, som programmet har sprunget over, og derved 
stadig burde være en del af beskeden.

Sidste del sker, fordi alle SMS’erne blev lagt sammen i en enkelt fil. Når programmet komprimerer en SMS, 
antallet af bits nødt til at gå op i 8, ellers vil programmet selv tilføje resten. Det betyder derfor, at 
fordi alle beskederne samles i en fil, vil det kun være ved den sidste besked at dette sker, og dette er 
en estimering, af hvor meget det vil svare til.

Det ses også hvordan en SMS komprimeret med et bit-træ skabt ud fra SMS’er fungerer bedre end et skabt 
fra wikipedias artikler.

\begin{figure}[H]
\centering
\begin{tikzpicture}
\begin{axis}[xbar stacked,
legend style={legend columns=4,at={(0,-0.35)},anchor=north west,draw=none},
ytick={0,1,2},
axis y line*=none,
axis x line*=bottom,
tick label style={font=\footnotesize},
legend style={font=\footnotesize},
label style={font=\footnotesize},
xtick={0,10,20,30,40,50,60,70,80,90,100},
width=.8\textwidth,
height=5cm,
bar width=5mm,
xlabel={Størrelse i procent},
yticklabels={WIKI bittræ, GSM bittræ, Original besked},
xmin=0,
xmax=100,
area legend,
enlarge y limits=0.6,
xticklabel=$\pgfmathprintnumber{\tick}$\,\%,
]
% Filstørrelse
\addplot[Sapphire,fill=Sapphire] coordinates
{(60.37,0) (58.34,1) (100,2)};
% Ubrugte tegn
\addplot[ForestLight,fill=ForestLight] coordinates
{(2,0) (2,1) (0,2)};
% Fejl i bit
\addplot[ForestGreen,fill=ForestGreen] coordinates
{(1.1,0) (1.1,1) (0,2)};
\legend{Filstørrelse,Ubrugte tegn, Fejl i bit}
\end{axis}
\end{tikzpicture}
\caption{Filstørrelse i forhold til original størrelse}
\label{resultatgraf}
\end{figure}


Som også ses på /ref{resultatgraf} er det gennemsnitligt lykkedes at komprimere en besked op til 42\% ved 
hjælp af det statiske bit-træ skabt ud fra SMS’er.
Her skal det så også pointeres at dette bit-træ er skabt ud fra disse SMS’er, så denne er netop optimeret 
til denne samling af SMS’er, hvorimod en ny samling, vil være en smule anderledes.

Ved at udregne ligningen X tegn * 0,6 = 160 tegn, hvor X er det antal tegn, som det er muligt 
gennemsnitligt at sende komprimeret. 0,6 er de 60\%, som det gennemsnitligt er muligt at mindske filens 
størrelse til. De 160 tegn, er det antal tegn, som der kan sendes med en SMS-besked.
Herved er det muligt at skrive op til 266 tegn i en enkelt SMS.
