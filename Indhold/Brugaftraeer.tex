I afsnittet \ref{huffman_traer}, ses der på hvilke metoder, der kan benyttes for at skabe de binære træer, der skal bruges ved dekodning af de sendte komprimerede beskeder.

Der blev nævnt to metoder. Den ene kaldt statisk, hvor der benyttes det samme binære træ for generel komprimering af tekst, mens den anden kaldt dynamisk, bliver lavet i forbindelse med hver enkelt komprimering.

Dette betyder derfor at komprimeringer hvor der bruges dynamiske træer, kræver at det binære træ, sendes sammen med den komprimerede besked. Hvorimod, ved brug af et statisk træ, er muligt at have det binære træ ved modtageren, af den komprimerede besked, på forhånd.

Ved brug af Huffman komprimering af korte beskeder, som ved sms’er, er det derfor mest hensigtsmæssigt at bruge statiske træer, da et medsendt dynamisk træ, kan få beskeden til at fylde mere end originalt.
