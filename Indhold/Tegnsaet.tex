For at kunne komprimere en besked er det vigtigt at kende til teknologien bag sms’er.

Den teknologi som anvendes i moderne telefoner hedder GSM (Global System for Mobile Communications), som er en 2G standard.# Denne teknologi gør det muligt at benytte sig af sms’er.\cite{GSM_term}

For at et computersystem skal have muligheden for at kunne printe tegn til skærmen, er det nødvendig at repræsentere disse tegn med hver sit tal. Disse tal har man bestemt i en standard, som betyder at alle skal benytte de samme tal, for de samme tegn, og derved gøre det lettere for programmørerne af softwaren der benytter disse tegn. Den mest brugte standard indenfor tegnsæt, som dette kaldes, er unicode.\cite{UNICODE_standard}

I mobiler der gør brug af GSM, kan der til sms’er, benyttes et tegnsæt kaldt GSM 03.38. Dette tegnsæt kan kodes i en række alfabeter, hvor standard alfabetet GSM 7 bit er et krav ved skabelse af mobiltelefoner.\cite{GSM_7_bit]

Nedenunder ses et skema af GSM 7 bit alfabetet: