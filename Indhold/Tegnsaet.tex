For at kunne komprimere en besked er det vigtigt at kende til teknologien bag SMS'er.

Den teknologi, der anvendes i moderne telefoner, hedder GSM (Global System for Mobile Communications) og er en 2G standard. Denne teknologi gør det muligt at benytte sig af SMS'er.\cite{GSM_term}

For at et computersystem skal have muligheden for at kunne printe tegn til skærmen, er det nødvendig at repræsentere disse tegn med hver sit tal. Disse tal har man bestemt i en standard, som betyder at alle skal benytte de samme tal, for de samme tegn, og derved gøre det lettere for programmørerne af software, der benytter disse tegn. Den mest brugte standard indenfor tegnsæt, som dette kaldes, er Unicode.\cite{UNICODE_standard}

I mobiltelefoner, der gør brug af GSM, kan der til SMS'er, benyttes et tegnsæt kaldet GSM 03.38. Dette tegnsæt kan kodes i en række alfabeter, hvor standard alfabetet GSM 7 bit, er det mest benyttet.
\\
\emph {Se skema af GSM 7 bit alfabetet i  bilag på figur \ref{tegnsaet} og figur \ref{tegnsaet2}}
\\
I dette skema symboliserer den vandrette linje, over den stiplede linje, det første ciffer, mens den lodrette kolonne er et andet ciffer. F.eks hvis man skal printe tegnet A, har den en talværdi på 42. Ved b1 til b7 vises også den binære talværdi.
På figur \ref{tegnsaet} ses ved 0x1b tegnet 1), der er et escape tegn til den ekstra tabel, som ses på figur \ref{tegnsaet2}. På figur \ref{tegnsaet2} ses det samme tegn, som her er et escape tegn som er reserveret hvis en ny ekstra tabel introduceres. Tegnet 3) er et form feed tegn, som er et escape tegn der starter en ny side.\cite{GSM_7_bit}
