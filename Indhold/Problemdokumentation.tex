Når det kommer til SMS-beskeder, så er der en grænse på hvor mange tegn der kan være i en enkelt besked. For tegn inkluderet i tegnsættet GSM 7-bit ligger begrænsningen på 160 tegn. Begrænsningen ændrer sig fra tegnsæt til tegnsæt. For eksempel har det kinesiske alfabet en tegnbegrænsning på 70 tegn\cite{Pro_1}. Normalt vil en besked som fylder mere end sin tegnbegrænsning blive delt op i to separate beskeder, hvis afsenderen af beskeden ikke selv gør det, hvilket kommer til at betyde dobbelt SMS-takst. Med denne begrænsning i tankerne kommer spørgsmålet: Hvor stor betydning har dette problem, og er det overhovedet værd at kigge nærmere på? 

Erhvervs priserne for at sende en SMS inden for Norden og Eurozonen er betydeligt billigere, end hvis man sendte til eller fra et europæisk land ikke inde under EU, og når man sender til eller fra lande udenfor Europa, så bliver det kun dyrere og dyrere. Et internationalt firma som udnytter SMS til intern kommunikation eller andet, kan ende med at bruge mange penge på deres telefonregninger. Tilbage i 2009/2010 begyndte de forskellige telefonselskaber at hæve prisen på afsendelse af beskeder til udlandet. TDC's pris, for eksempel, gik fra at være på 2,40 kr. til at koste 3,20 kr. per SMS\cite{Pro_2}. Nedenstående tabel viser Telenors SMS-takst samt minutpris for erhverv, ved at sende beskeder til Danmark, men prisen er stadig den samme fra Danmark til udlandet\cite{Pro_3_1}.

%\noindent
\begin{table}[H]
\begin{center}
\begin{tabular}{ | l | r |}
    \hline
    \cellcolor{ForestGreen} &  \cellcolor{ForestGreen}\color{white}{\textbf{Sende/Modtage SMS}}\\[2ex] \hline
    \textbf{Norden} & 0,66 kr./sms \\ \hline
    \textbf{EU} & 0,66 kr./sms \\ \hline
    \textbf{Øvrige Europa} & 3,20 kr./sms \\ \hline
    \textbf{Verden 1} & 3,20 kr./sms \\ \hline
    \textbf{Verden 2} & 3,20 kr./sms \\ \hline
    \textbf{Skibe m. MCP-dækning} & 3,20 kr./sms \\ \hline
\end{tabular} 
\caption{Tabel over SMS-Priser fra Telenor ~\cite{Pro_3_2}}
\end{center}
\end{table}

Ligeledes er SMS-taksterne for private hen over landegrænserne også på den dyre siden. I det private strækker priserne sig fra 3's pris pr. SMS på 2,50 kr\cite{Pro_4} til Telia's pris pr. SMS på 4,00 kr\cite{Pro_5}. Uanset om man er privat eller erhvervsdrivende, så vil man gerne være sparsomme, med antallet af beskeder man sender over landegrænserne, og derudfra gøre god brug af sine 160 tegn, sådan at man undgår dobbelt SMS-takst ved at beskeden bliver delt i to.

Statistikkerne viser at der i 2011 blev sent omkring 12,3 milliarder SMS-beskeder i Danmark alene, et fald fra forrige år som lå på 13 millarder, men det viser, at der stadigvæk er er højt forbrug af SMS-beskeder. Derudover så steg den mobile datatrafik fra 15 milliarder MB i 2010 til 26 milliarder MB i 2011\cite{Pro_6}. En løsning som komprimerer beskeder, kunne også hjælpe til at dæmpe belastningen på det mobile datatrafik netværk.

Derfor vil en  datalogisk løsning, som gør det lettere at sende beskeder, uden at man skal bekymre som om, hvorvidt ens besked har mere end de begrænsede 160 tegn, være aktuelt. Sådan en løsning kan både gøre det mere bekvemt for brugeren at bruge SMS'er, og i det lange løb sparer brugeren penge.