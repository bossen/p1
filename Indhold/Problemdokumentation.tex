N�r det kommer til SMS beskeder, s� er der en gr�nse p� hvor mange tegn der kan v�re i en enkelt besked. For det latinske alfabet ligger begr�nsningen p� 160 tegn. Begr�nsningen �ndrer sig fra tegns�t til tegns�t. For eksempel har det kinesiske alfabet en tegnbegr�nsning p� 70 tegn\cite{Pro_1}. Normalt vil en besked som fylder mere end sin tegnbegr�nsning blive delt op i to separate beskeder, hvis afsenderen af beskeden ikke selv g�r det, hvilket kommer til at betyde dobbelt SMS takst. Med denne begr�nsning i tankerne kommer sp�rgsm�let: Hvor betydeligt er dette problem, og er det overhovedet v�rd at kigge n�rmere p�?
Erhvervs priserne for at sende en SMS inden for Norden og Eurozonen er betydeligt billigere end hvis man sendte til eller fra et Europa land ikke inde under EU, og n�r man sender til eller fra lande udenfor Europa s� bliver det kun dyrere og dyrere. Et internationalt firma som udnytter SMS til intern kommunikation eller andet kan ende med at bruge mange penge p� deres telefonregninger. Tilbage i 2009/2010 begyndte de forskellige telefonselskaber at h�ve prisen p� afsendelse af beskeder til udlandet. TDC's pris, for eksempel, gik fra at v�re p� 2,40 kr. til at koste 3,20 kr. per SMS\cite{Pro_2}. Nedenst�ende tabel viser Telenors SMS takst samt minutpris for erhverv ved at sende beskeder til Danmark, men prisen er stadig den sammen den fra Danmark til udlandet\cite{Pro_3}.

%\noindent
\begin{table}[H]
\begin{center}
\begin{tabular}{ | l | r |}
    \hline
    \cellcolor{ForestGreen} &  \cellcolor{ForestGreen}\color{white}{\textbf{Sende/Modtage SMS}}\\[2ex] \hline
    \textbf{Norden} & 0,66 kr./sms \\ \hline
    \textbf{EU} & 0,66 kr./sms \\ \hline
    \textbf{�vrige Europa} & 3,20 kr./sms \\ \hline
    \textbf{Verden 1} & 3,20 kr./sms \\ \hline
    \textbf{Verden 2} & 3,20 kr./sms \\ \hline
    \textbf{Skibe m. MCP-d�kning} & 3,20 kr./sms \\ \hline
\end{tabular} 
\caption{Tabel over SMS-Priser fra Telenor ~\cite{Pro_3}}
\end{center}
\end{table}

Ligeledes er priserne for private hen over landegr�nserne heller ikke noget at prale af. I det private str�kker priserne sig fra 3's pris pr. SMS p� 2,50 kr\cite{Pro_4} til Telia's pris pr. SMS p� 4,00 kr\cite{Pro_5}. Uanset om man er privat eller erhvervsdrivende s� vil man gerne v�re sparsomme med antallet af beskeder man sender over landegr�nserne, og derudfra g�re god brug af sine 160 tegn s�dan at man undg�r dobbelt SMS takst ved at beskeden bliver delt i to.
Derfor vil en eller anden datalogisk l�sning, som g�r det lettere at sende beskede, uden at man skal bekymre som om hvorvidt ens besked har mere end de begr�nsede 160 tegn, v�re aktuelt. S�dan en l�sning kan b�de g�re det mere bekvemt for brugeren at bruge SMS'er, og i det lange l�b sparer brugeren penge.