Alle tre modeller har sine fordele og ulemper. I dette afsnit der blive delkonkluderet på hele af afsnittet, og hver model vil blive sammenlignet med hinanden, og til resten af projektet, for at finde den model som bedst kan bruges til at besvare problemformuleringen omkring SMS beskeder.

Først er der den statiske model. Som sagt, så kræver den statiske model ikke særlig meget regnekraft. Derudover så er den også let at implementere fordi det er, som navnet hentyder, et statisk træ og er derfor altid ens. Ulempen ved den statiske model er at den ikke altid kan gøre optimalt brug af Huffman coding. Den tekst der skal komprimeres stemmer ikke altid overens med forekomst ratioen og kan derfor ikke gøre perfekt brug af Huffman. På dette punkt er både den dynamiske og den adaptive model bedre som kan lave sine egne træer alt efter hvad der skal komprimeres.

Med det kommer den dynamiske model. Af de tre modeller gør den dynamiske model bedst brug af komprimeringskraften ved Huffman coding idet at den laver et helt nyt træ ud fra det tekststykke den skal komprimere. Den dynamiske model har dog en del implementerings ulemper. Som sagt, så skal der sendes en manual med, som optager plads, for at kunne dekomprimere tekststykket igen. Derudover så kræver den dynamiske model også en del ekstra regnekraft på enheden for at generer et nyt træ hver gang der skal sendes en besked.

Den sidste model er den adaptive model. Fordelen ved den adaptive model er at den gør god brug af Huffman coding som den dynamiske model, og behøver ikke at sende noget ekstra med for at sikre modtageren kan dekomprimere beskeden, som den statiske model.Ulempen ved den adaptive model er at den ikke altid kan gøre optimalt brug af Huffman coding idet at træet først opdateres efter den har komprimeret og dekomprimeret et tekststykke. Derudover så er det også den model som kræver mest arbejde for at virke, og tager mest regnekraft på enheden.

Til sidst kommer spørgsmålet så igen; Hvilken model vil være bedst til at bruge i besvarelsen af projektets problemformulering? For det første så kræver både den dynamiske og den adaptive model langt mere regnekraft, i forhold til den statiske, for at kunne fungere. Dette kan give problemer idet programmet skal kunne køre på mobiltelefoner. Det er også blevet vist at komprimering med Huffman er meget effektivt, så tabet i komprimering ved at bruge den statiske og den adaptive model er ikke alt for væsentlig. Derudover så bliver komprimeringen mere effektiv, alt efter hvor lang tekststykket er, og målet med projektet var at kunne komprimere SMS beskeder som overstiger tegnbegrænsningen. Den statiske og den adaptive model behøver heller ikke at bruge ekstra plads, som den dynamiske gør, for at sikre at beskeden kan blive dekomprimeret igen. Ud fra disse punkter, samt sammenligningen og resten af afsnittet, blev det bestemt at løsningen skulle laves ved hjælp af et statisk Huffman træ.