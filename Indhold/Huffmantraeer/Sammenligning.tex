Alle tre modeller har sine fordele og ulemper. I dette afsnit vil de tre forskellige type træer sammenlignes og det diskuteres hvilken type der vil være bedst at bruge til løsningen af projektet.

Først er der den statiske model. Som sagt, så kræver den statiske model ikke særlig meget regnekraft. Derudover så er den også let at implementere fordi den, som navnet antyder, er statisk og derfor altid er ens. Ulempen ved den statiske model er, at den ikke altid kan gøre optimalt brug af Huffman algoritmen, fordi den tekst der skal komprimeres ikke altid stemmer overens med forekomst raten. På dette punkt er både den dynamiske og den adaptive model bedre, fordi de kan lave sine egne træer alt efter hvad, der skal komprimeres.

Af de tre modeller gør den dynamiske model bedst brug af komprimeringskraften ved Huffman algoritmen, idet den laver et helt nyt træ ud fra det tekststykke, den skal komprimere. Den dynamiske model har dog en del implementerings ulemper. For det første så skal det dynamiske træ sendes med for at tekststykket kan dekomprimeres igen, dette kræver plads. Derudover så kræver den dynamiske model også en del ekstra regnekraft, fordi der skal generes et nyt træ hver gang der skal komprimeres et stykke tekst.

Fordelen ved det adaptive træ er at det ikke skal sendes med for at modtageren kan dekomprimere beskeden. Ulempen er, at den ikke altid kan gøre optimalt brug af Huffman algoritmen, idet at træet først opdateres efter den har komprimeret og dekomprimeret et tekststykke. Derudover så er det også den model som kræver mest arbejde for at virke, og tager mest regnekraft på enheden.

Hvilken model vil så være bedst til at løse projektets problemformulering? For det første så kræver både den dynamiske og den adaptive model langt mere regnekraft, i forhold til den statiske. Dette kan give problemer idet programmet skal kunne køre på mobiltelefoner. Derudover så behøver hverken den statiske eller den adaptive model at medsende træet, som kræver ekstra plads, for at beskeden kan dekomprimeres. Problemet, der skal løses, handler netop om at der skal sendes så lidt tegn som muligt pr SMS, samtidigt skal programmet køres på mobiltelefoner, som ikke nødvendigvis har ubegrænset regnekraft. Herved er det statiske træ nok den bedste model at bruge til løsningen. Vi vil dog både lave en løsning med det statiske træ og en løsning med det dynamiske træ, for at kunne sammenligne disse og prøve at konkludere på hvilken model, bedst komprimere SMS-beskeder.