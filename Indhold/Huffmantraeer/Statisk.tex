Den første metode man kan bruge til at generere et Huffman træ, er den statiske metode. Et statisk Huffman træ bliver lavet ud fra formodede forekomster af tegn i et stykke tekst. For eksempel hvis man kigger generelt på det engelske sprog, så forekommer tegnet ’e’ mest, og vil derfor blive placeret øverst i det statiske Huffman træ, mens tegn som ’z’ og ’x’ vil blive placeret nederst\cite{Hufftree_2}. Ligeledes kan man se i afsnit \ref{tegnForekomst} hvor tit tegn forekommer ud fra dette projekts undersøgelser. Den statiske opbygning af et Huffman træ er i øvrigt standard metoden.

Et statisk Huffman træ virker på alle stykker tekst, især længere stykker af tekst som for eksempel artikler. Til gengæld kan den ende med at ikke gøre fuld brug af komprimeringskraften ved Huffman coding, når det handler om mindre beskeder, fordi det ikke altid går op med normalen for tegn forekomster i et sprog. Statiske Huffman træer virker bedre, i takt med at den tekst som skal komprimeres bliver større, og bliver mere ineffektiv, når teksten bliver mindre. Det er ikke sandsynligt, at den komprimerede tekst vil fylde mere, end hvis den ikke var komprimeret idet at Huffman komprimering er så effektivt som det er. \cite{Hufftree_3}

Fordele ved et statisk træ, er at det let kan give gode resultater for større stykker af tekst og behøver ikke at sende noget yderligt i beskeden udover bitmønstret. Et statisk træ er også hurtigt, idet der ikke er behov for at generere et nyt træ for hvert eneste stykke tekst, som der er behov for med de to kommende metoder. Statisk kan derfor have en fordel, fordi den ikke behøver ligeså meget regnekraft på enheder som mobiltelefoner, som muligvis kan være begrænset på dette område.
