Hvis vi skulle lave projeket igen, ville vi forsøge at indsamle data om interessenterne for løsningen. Det ville eksempelvis være interessant at vide hvor mange SMS'er de sender, både indenrigs og udenrigs. Derudover ville det selvfølgelig være vigtig at vide hvor mange af deres SMS-beskeder, der overstiger 160 tegn. Til at indsamle denne kvantitative data, ville en spørgeskema undersøgelse være en god mulighed. Ved at have denne data, kunne man regne på den faktiske besparelse på SMS-forbrug for et firma eller privat personer, hvis de implementerede vores software. Herved kunne vi vurdere, om programmet ville have en effekt af betydning. 

Derudover kunne det også være relevant at kigge på om vores software ville påvirke teleselskaberne og hvorvidt det ville være positivt eller negativt. Spørgsmålet er om de ville miste penge ved at deres kunder sender færre SMS'er, eller ville de selv kunne spare penge hvis de implementerede vores software i deres netværk.
Derudover kunne man også undersøge hvor man ellers med fordel kunne bruge datakomprimering. Det kunne eksempelvis være når man skal lagre filer og ved generel internet trafik. Derved kunne vi finde ud af om vores software kunne være relevant i andre sammenhænge.

I fremtiden kunne man også overveje, hvorvidt PPM kunne blive implementeret. PPM kan blive bygget ovenpå den allerede eksisterende Huffman komprimeringsalgoritme, og derved gøre løsningen langt bedre. Derudover skal der også laves beregninger på, hvor meget regnekraft og plads man har tilgængelige på en mobiltelefon eller smartphone. Det er muligt, at især smartphones har nået et punkt, hvor de sagtens kan håndtere tungere programmer, som derfor åbner op for mulige løsninger ved at bruge PPM og derudover også bruge dynamiske eller adaptive Huffman træer.