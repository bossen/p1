Entropikodning er en lossless/tabsfri datakomprimeringsmetode. Tabsfri, betyder at der ikke går nogen information tabt, ved at komprimere datamængden. Modsat har vi lossy/tabsgivende komprimering, som fx MP3, og JPEG. Entropopikodning går ud på, at få en given datamænde til at benytte et minimalt antal bit. Dette kan opnås ved at kigge på hyppigheden af de forskellige tegn i datamængden der skal komprimeres, og give de oftest fremkommende symboler få bits, og de mere sjældne symboler flere bits. Formålet er, at få det gennemsnitlige antal bits pr. symbol(middelkodelængden) ned. Den teoretiske nedre grænse for middelkodelængden kaldes datamængdens entropi. \cite{entro1}
Der findes flere forskellige entropikodningsmetoder, og et par eksempler er "Huffman kodning" og "arithmetic coding".