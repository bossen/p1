PPM blev udviklet af John Cleary og Ian Witten. De beskrev metoden i en artikel de udgav i 1984; "Data Compression Using Adaptive Coding and Partial String Matching" \cite{Cleary84datacompression}.

PPM er en tabsfri komprimeringsmetode, der er blandt de bedste til at komprimere tekst. PPM står for "Prediction by partial matching" (Forudsigelse ved delvis matching), hvilket også fortæller lidt om hvordan komprimeringsmetoden fungerer. PPM forsøger at forudsige, hvad det næste tegn i datamængden vil være, ved at kigge på den ikke komprimerede data, og kigge efter om der findes en lignende sammensætning af tegn, og hvor ofte forskellige tegn forekommer. 

Algoritme \ref{pseudo_ppm} viser en simpel implementering i pseudokode, der bearbejder tegn for tegn, og forsøger at lave nogle sammenhænge, ud fra om noget går igen. \cite{ppm_stringology}

\begin{algorithm}[H]
 \SetAlgoLined
% \KwData{this text}
% \KwResult{how to write algorithm with \LaTeX2e }
 \While{ikke sidste tegn}{ %not last character
   læsSymbol()\; %readSymbol()
   forkort kontekst\; %shorten context
   \While{sammenhæng ikke fundet og sammenhæng længde ikke er lig -1}{ %context not found and context length not -1
       output(sekvens)\; %escape sequence
       forkort sammenhæng\; %shorten context
       }
   output(tegn)\;
   \While{kontekst længde ikke er -1}{ %context length not -1
      tæl tegntæller op\; %increase count of character (create node if nonexistant)
      forkort kontekst\; %shorten context
      }
      }
\caption{Pseudokode af PPM komprimering \cite{ppm_stringology}}
\label{pseudo_ppm}
\end{algorithm}