PPM blev udviklet af John Cleary og Ian Witten. De beskrev metoden i en artikel de udgav i 1984; "Data Compression Using Adaptive Coding and Partial String Matching" \cite{Cleary84datacompression}.

PPM er en tabsfri komprimeringsmetode, der er blandt de bedste til at komprimere tekst. PPM st�r for "Prediction by partial matching" (Forudsigelse ved delvis matching), hvilket ogs� fort�ller lidt om hvordan komprimeringsmetoden fungerer. PPM fors�ger at forudsige, hvad det n�ste bogstav i datam�ngden vil v�re, ved at kigge p� den ikke komprimerede data, og kigge efter om der findes en lignende sammens�tning af tegn, og hvor ofte forskellige tegn forekommer. 

Algoritme \ref{pseudo_ppm} viser en simpel implementering i pseudokode, der bearbejder tegn for tegn, og fors�ger at lave nogle sammenh�nge, ud fra om noget g�r igen. \cite{ppm_stringology}

\begin{algorithm}[H]
 \SetAlgoLined
% \KwData{this text}
% \KwResult{how to write algorithm with \LaTeX2e }
 \While{not last character}{
   readSymbol()\;
   shorten context\;
   \While{context not found and context length not -1}{
       output(escape sequence)\;
       shorten context\;
       }
   output(character)\;
   \While{context length not -1}{
      increase count of character (create node if nonexistant)\;
      shorten context\;
      }
      }
\caption{Pseudokode af PPM komprimering \cite{ppm_stringology}}
\label{pseudo_ppm}
\end{algorithm}