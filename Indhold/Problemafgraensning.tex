Der er en række forskellige ting og områder som man kan komme omkring når det kommer til komprimering af beskeder. Det første er hvilke internet-servicer samt teknologier som har et begrænset antal af tegn pr. besked. Her kan SMS indgå som har sin begrænsning på 160 tegn når man bruger det latinske alfabet\cite{Pro_1}. Derudover kan en internetservice som for eksempel Twitter også indgå, som har en tegn begrænsning på 140 tegn\cite{pro_af1}. Fremover vil der ikke blive arbejdet med Twitter når vi skal udarbejde et produkt eller anden eventuel løsning på komprimering af tegn begrænset beskeder, da Twitters formål til at starte med var korte og smertefrie bidder af information delt over internettet, og ikke lange blogs og artikler. Denne holdning er folkene bag Twitter meget konsekvente med\cite{pro_af2}, og det er derfor der vil blive arbejde med SMS i dette projekt.

Nu hvor at valget om SMS eller Twitter er på plads, så kommer spørgsmålet om hvorvidt der skal arbejdes med smartphones i tankerne, eller almindelige mobiltelefoner. Uanset om det er smartphones eller mobiltelefoner, løsningen bliver lavet til, så vil den nok ikke blive fuldt implementeret, da det ikke ligger inden for projektets omfang at kigge helt ned og pille rundt i dybden på telefonernes hardware. Smartphones har dog den fordel at de har lettere ved at implementere apps uden alt for meget besvær, og kan derfor muligvis blive brugt til at teste om vores løsning virker. Derfor vil der fremover i projektet blive arbejdet med smartphones.

Til sidst skal der også tages højde for hvilke tegnsæt vores løsning skal kunne dække. Altså skal vores løsning være i stand til at komprimere beskeder lavet med tegn fra det latinske alfabet, det kinesiske, de slaviske alfabeter eller andre alfabeter, samt specialtegn og andre mærkelige tegn som Æ, Ø, Å, ß osv. Til at starte med vil der være en fokus på standard tegnsæt. Standard tegnsæt kan blive fundet bedre beskrevet i tegnsæt sektionen. Som projektet skrider frem kan andre alfabeter og diverse tegn blive inkluderet i løsningen hvis tiden tillader.