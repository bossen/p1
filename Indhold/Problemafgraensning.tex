Der er en r�kke forskellige ting og omr�der som man kan komme omkring n�r det kommer til komprimering af beskeder. Det f�rste er hvilke internet servicer og eller ydelser, samt teknologier, som har et begr�nset antal af tegn pr. besked. Her kan SMS indg� som har sin begr�nsning p� 160 tegn n�r man bruger det latinske alfabet\cite{Pro_1}. Derudover kan en internetservice som for eksempel Twitter ogs� indg�, som har en tegn begr�nsning p� 140 tegn\cite{pro_af1}. Fremover vil der ikke blive arbejdet med Twitter n�r vi skal udarbejde et produkt eller anden eventuel l�sning p� komprimering af tegn begr�nset beskeder, da Twitters form�l til at starte med var korte og smertefrie bidder af information delt over internettet, og ikke lange blogs og artikler. Denne holdning er folkene bag twitter meget konsekvente med\cite{pro_af2}, og det er derfor der vil blive arbejde med SMS i dette projekt.

Nu n�r at valget om SMS eller Twitter er p� plads, s� kommer sp�rgsm�let om hvorvidt der skal arbejdes med smartphones i tankerne, eller almindelige mobiltelefoner. Uanset om det er smartphones eller mobiltelefoner, l�sningen bliver lavet til, s� vil den nok ikke blive fuldt implementeret, da det ikke ligger inden for projektets omfang at kigge helt ned og pille rundt i dybden p� telefonernes hardware. Smartphones har dog den fordel at de har lettere ved at implementere apps uden alt for meget besv�r, og kan derfor muligvis blive brugt til at teste om vores l�sning virker. Derfor vil der fremover i projektet blive arbejdet med smartphones.

Til sidst skal der ogs� tages h�jde for hvilke tegns�t vores l�sning skal kunne d�kke. Alts� skal l�sningen kunne komprimere beskeder skrevet med det latinske alfabet, det kinesiske, de slaviske alfabeter osv., samt specialtegn og andre m�rkelige tegn som �, �, �, � osv. Til at starte med vil der v�re en fokus p� standard tegns�t. Standard tegns�t kan blive fundet bedre beskrevet i tegns�t sektionen. Som projektet skrider frem kan andre alfabeter og diverse tegn blive inkluderet i l�sningen hvis tiden tillader.