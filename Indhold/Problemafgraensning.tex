Dette afsnit kommer omkring nogen af de valg der blev lavet for at indskrænke projektets problemfelt. For det første er der forskellige tjenester og teknologier, som giver muglihed for at sende en kort besked, men med et begrænset antal tegn pr. besked. Eksempelvis er der SMS-beskeder med en begrænsning på 160 tegn når man bruger tegn fra tegnsættet GSM 7-bit\cite{Pro_1} og der er også internettjenester som for eksempel Twitter, som har en tegnbegrænsning på 140 tegn\cite{pro_af1}. Twitters formål har fra starten af, været at give mulighed for at sende korte og smertefrie bidder af information over internettet, og ikke lange blogs og artikler. Denne holdning er folkene bag Twitter meget konsekvente med\cite{pro_af2}. Derudover så er det også gratis at gøre brug af Twitter og derfor er det ikke ligeså væsentligt som SMS, som koster penge. Derfor har vi valgt ikke at arbejde med Twitter. Istedet vil projektet blive begrænset til at handle om SMS-beskeder.

Nu hvor at valget om SMS eller Twitter er på plads, så kommer spørgsmålet om hvorvidt der skal arbejdes med smartphones eller almindelige mobiltelefoner. Smartphones har den fordel at de kan implementere applikation uden alt for meget besvær, hvorimod på almindelige telefoner er det meget mere besværligt at installere programmer. Statistikkerne viser at flere og flere begynder at få smartphones\cite{pro_af3}. I det sidste kvartal af 2011 blev der solgt over 37 millioner iPhones, Apple's smartphone, i hele verdenen, som er højere end antallet af børn født i den samme periode\cite{pro_af4}. Derudover så bliver der også vist at brugen af hjemme computere er dallende i det at adgang til internettet også er tilgængeligt gennem smartphones, som derved gør det muligt for en person at være på internettet hvor man ellers ikke ville have tilgang til en almindelig computer\cite{pro_af3}. Dette kan betyde at flere personer bruger SMS fordi de bruger deres mobile maskiner mere. Dog kan det også betyde at flere mennesker bruger e-mail i stedet for SMS fordi de alligevel har adgang til internettet. 

Ud fra dette vil projektet yderligere blive begrænset til ikke at prøve at implementere programmet på almindelige mobiltelefoner. Derudover så er det også vanskeligt at implementere en applikation skrevet i C, som er et kræv for dette projekt, på en smartphone. Derfor vil løsningen være en prototype som kan fungerer på en computer. Løsningen skal kunne køre uden at forstyrrer brugeren alt for meget, både i form af plads og regnekraft programmet kræver, og skulle gerne kunne køre at sig selv. Hvis det er nødvendigt med yderlig bruger interaktion så skal der, selvfølgelig, foregå så brugervenligt som muligt.

Det sidste punkt er hvilke tegnsæt som løsningen skal være i stand til at komprimere og dekomprimere. Skal tegn fra det kyrilliske alfabet eller specialtegn som Æ, Ø, Å, ß være i stand til at blive komprimeret, for eksempel. Det er blevet bestemt at løsningen skal have implementeret GSM 7-bit tegnsættet, som er standard tegnsættet til SMS beskeder på mobiltelefoner og smartphone.
