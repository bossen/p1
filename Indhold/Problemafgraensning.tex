Vi vil i dette afsnit afgr�nse vores problemfelt, til noget vi kan arbejde med i dette projekt. For det f�rste er der forskellige tjenester og teknologier, som giver muglihed for at sende en kort besked, men med et begr�nset antal tegn pr. besked. Eksempelvis er der SMS-beskeder med en begr�nsning p� 160 tegn n�r man bruger det latinske alfabet\cite{Pro_1} og der er ogs� internettjenester som for eksempel Twitter, som har en tegnbegr�nsning p� 140 tegn\cite{pro_af1}. Twitters form�l har fra starten af, v�ret at give mulighed for at sende korte og smertefrie bidder af information over internettet, og ikke lange blogs og artikler. Denne holdning er folkene bag Twitter meget konsekvente med\cite{pro_af2}, og derfor vil vi ikke arbejde videre med Twitter. Istedet vil vi begr�nse vores projekt til at handle om SMS-beskeder.

Nu hvor at valget om SMS eller Twitter er p� plads, s� kommer sp�rgsm�let om hvorvidt der skal arbejdes med smartphones eller almindelige mobiltelefoner. Smartphones har den fordel at de kan implementere apps uden alt for meget besv�r, hvorimod at hvis l�sningen skulle bruges p� almindelige mobiltelefoner skal den pr�installeres. Derfor vil vi yderligere afgr�nse vores projekt til kun at arbejde med smartphones. Selvom vi v�lger at arbejde med smartphones s� vil vores l�sningen ikke blive fuldt implementeret, da vi i programmeringssproget C ikke har mulighed for at lave en rigtig app. Vi vil i stedet afpr�ve vores l�sning p� en computer.

Vi vil ydermere ikke arbejde med alle eksisterede tegns�t, da der eksempelvis b�de findes tegn fra det latinske alfabet, det kinesiske, de slaviske alfabeter, samt specialtegn som �, �, �, � osv. Til at starte med vil der v�re en fokus p� standard tegns�t. Standard tegns�t kan bliver beskrevet i afsnittet Tegns�t. Som projektet skrider frem kan andre alfabeter og diverse tegn blive inkluderet i l�sningen, hvis tiden tillader.