Dette afsnit kommer omkring nogen af de valg der blev lavet for at indskr�nke projektets problemfelt. For det f�rste er der forskellige tjenester og teknologier, som giver muglihed for at sende en kort besked, men med et begr�nset antal tegn pr. besked. Eksempelvis er der SMS-beskeder med en begr�nsning p� 160 tegn n�r man bruger det latinske alfabet\cite{Pro_1} og der er ogs� internettjenester som for eksempel Twitter, som har en tegnbegr�nsning p� 140 tegn\cite{pro_af1}. Twitters form�l har fra starten af, v�ret at give mulighed for at sende korte og smertefrie bidder af information over internettet, og ikke lange blogs og artikler. Denne holdning er folkene bag Twitter meget konsekvente med\cite{pro_af2}. Derudover s� er det ogs� gratis at g�re brug af Twitter og derfor er det ikke liges� v�sentligt som SMS, som koster penge. Derfor har vi valgt ikke at arbejde med Twitter. Istedet vil projektet blive begr�nset til at handle om SMS-beskeder.

Nu hvor at valget om SMS eller Twitter er p� plads, s� kommer sp�rgsm�let om hvorvidt der skal arbejdes med smartphones eller almindelige mobiltelefoner. Smartphones har den fordel at de kan implementere applikation uden alt for meget besv�r, hvorimod p� almindelige telefoner er det meget mere besv�rligt at installere programmer. Derfor vil projektet yderligere blive begr�nset til ikke at pr�ve at implementere programmet p� almindelige mobiltelefoner. Derudover s� er det ogs� vanskeligt at implementere en applikation skrevet i C, som er et kr�v for dette projekt, p� en Smartphone. Derfor vil l�sningen v�re en prototype som kan fungerer p� en computer.

Det sidste punkt er hvilke tegns�t som l�sningen skal v�re i stand til at komprimere og dekomprimere. Skal tegn fra det kyrilliske alfabet eller specialtegn som �, � �, � v�re i stand til at blive komprimeret, for eksempel. Til at starte med s� vil der v�re fokus p� at f� standard tegns�ttet ASCII implementeret i programmet. Derefter kan der blive arbejdet hen imod GSM 7-bit, som er standarden for mobiltelefoner.
