Til dette projekt skal der udarbejdes en løsning i form at et program, som kan komprimere en kort tekstbesked. Komprimeringen vil gøre tekstbeskeden mindre, det vil sige beskeden fylder færre bytes, og skulle gerne både gøre det hurtigere at sende beskeden, fordi den er mindre, men også gøre det muligt at sende en besked over en bestemt tegn begrænsning, som f.eks. de begrænsede 160 tegn ved brug af det latinske alfabet i en SMS. Beskeden skal derefter dekomprimeres hos modtageren, og derefter vise beskeden, som den så ud før den blev komprimeret. Denne proces skal ske, uden brugeren selv tager en direkte del i processen.\\

\textbf{Funktionelle Krav}
\begin {itemize}
	\item Skal både være i stand til at komprimere og dekomprimere automatisk.
	\item Skal være i stand til at skelne mellem hvorvidt den pågældende besked skal komprimeres eller dekomprimeres.
	\item Det er ikke forventet, at prototypen skal kunne køre på en mobil enhed, men det er forventet at programmet kan bruges på en computer.
\end{itemize}

\textbf{Ikke Funktionelle Krav}
\begin {itemize}
	\item Produktet skal afleveres sammen med den tilhørende rapport, og har en fælles deadline den 19 december 2012.
	\item Programmet skal skrives i programmeringssproget C.
\end{itemize}

\textbf{Løsningsmål}
\begin {itemize}
	\item Brugeren skal kunne gøre brug af programmet uden selv at tage direkte del i komprimerings-processen.
	\item Programmet skal køre lokalt, og ligeledes skal komprimeringen og dekomprimeringen også ske lokalt.
	\item Programmet skal implementeres og være brugbart.
\end{itemize}
