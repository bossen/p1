Til dette projekt skal der udarbejdes en løsning i form at et program, som kan komprimere en kort tekstbesked. Komprimeringen vil gøre tekstbeskeden mindre, det vil sige beskeden fylder færre bytes, og skulle gerne både gøre det hurtigere at sende beskeden, fordi den er mindre, men også gøre det muligt at sende en besked over en bestemt tegn begrænsning, som f.eks. de begrænsede 160 tegn ved brug af det latinske alfabet i en SMS. Beskeden skal derefter dekomprimeres hos modtageren, og derefter vise beskeden, som den så ud før den blev komprimeret. Denne proces skal ske, uden brugeren selv tager en direkte del i processen.

\begin {itemize}
\item Funktionelle Krav til programmet
\subitem Skal både være i stand til at komprimere og dekomprimere automatisk.
\subitem Skal være i stand til at skelne mellem hvorvidt den pågældende besked skal komprimeres eller dekomprimeres.
\subitem Det er ikke forventet, at prototypen skal kunne køre på en mobil enhed, men det er forventet at programmet kan bruges på en computer.

\item Ikke Funktionelle Krav
\subitem Produktet skal afleveres sammen med den tilhørende rapport, og har en fælles deadline den 19 december 2012.
\subitem Programmet skal skrives i programmeringssproget C.

\item Løsningsmål
\subitem Brugeren skal kunne gøre brug af programmet uden selv at tage direkte del i komprimerings-processen.
\subitem Programmet skal køre lokalt, og ligeledes skal komprimeringen og dekomprimeringen også ske lokalt.
\subitem Programmet skal implementeres og være brugbart.
\end{itemize}
