Dette projekt handler om tekstkomprimering og om hvordan man kan bruge dette i forbindelse med SMS-beskeder, som har en øvre grænse for hvor mange tegn man kan sende pr. besked. Den første del af vores problemformulering lyder;
\emph{Hvordan kan man spare forbrugeren for dobbelt SMS-takst, ved brug af et komprimeringsprogram?}

Til at besvarer dette spørgsmål, hvar vi udviklet et program, som kan komprimere en tekstbesked ved brug af Huffman-algoritmen. Som det ses i afsnittet RESULTATER er det lykkedes af komprimere bedskeder med over 50 procent, hvilket betyder at det, med brug af programmet, er muligt at sende to SMS-beskeder som én. På den måde vil brugeren kunne spare SMS-takst, hver gang der skal sendes en SMS på over 160 tegn.
Vi kan derfor konkludere at man godt kan sparer brugeren for dobblet SMS-takst. Den samlede besparelse på SMS-forbruget, kommer både an på hvor mange SMS'er der afsendes og ligeledes an på hvor mange af dem som fylder mere end 160 tegn. 
Næste del af vores problemformulering lyder;
\emph{Hvordan kan man undgå at programmet bliver en belastning for brugeren?} 
Vi kan konkludere 

