I dette afsnit bliver der kigget tilbage på problemformuleringen og ser på hvordan spørgsmålene er blevet besvaret, både hvis det viste sig at være positivt eller negativt.

Den første del af problemformuleringen lyder;\\
\emph{Hvordan kan man spare forbrugeren for dobbelt SMS-takst, ved brug af et komprimeringsprogram?}
 
Til at besvare dette spørgsmål blev der udviklet et program, som kan komprimere en tekstbesked ved brug af Huffman-algoritmen. Huffman komprimeringsalgoritmen blev valgt til udviklingen af løsningen, fordi den passede bedst overens med projektets rammer, samt den platform som programmet var rettet imod. Som det ses i afsnittet \ref{resultater} er det lykkedes at komprimere beskeder med over 30 til 40 procent, hvilket betyder at det, med brug af programmet, er muligt at sende to SMS-beskeder som én. På den måde vil brugeren kunne spare SMS-takst, hver gang der skal sendes en SMS på over 160 tegn.
 
Derudover er der også blevet dokumenteret for at problemstillingen er mulig at løse. Der blev vist, at brugen af SMS er fortsat stor, selvom andre mobile enheder, som har tilgang til internet, heriblandt smartphones, med mulighed for at sende e-mails uden tegn begrænsning, stadigvæk overtager mere og mere af markedet. Der blev også vist, at især SMS-taksten til og fra Danmark og bestemte lande er høj i forhold til SMS-taksten mellem Danmark og andre bestemte lande. Derudover så blev der også vist, at SMS-taksten begyndte at stige i nyere tid.
 
Vi kan derfor konkludere at man kan spare brugeren penge ved at komprimere en SMS besked som overstiger tegnbegrænsningen, og dermed gøre en besked der normalt ville tage dobbelt SMS-takst til enkelt SMS-takst. Derudover, måske lidt mindre væsentligt, så kan man også sænke belastningen på det mobile datatrafik netværk, idet Huffman komprimering gør størrelsen på beskeden mindre. Den samlede besparelse på SMS-forbruget, kommer både an på hvor mange SMS'er der sendes og ligeledes an på hvor mange af dem der fylder mere end 160 tegn.

Næste del af problemformuleringen lyder;\\
\emph{Hvordan kan man undgå at programmet bliver en belastning for brugeren?} 

Til at besvarer dette spørgsmål så blev det kigget på de forskellige fordele og ulemper der er ved at bruge forskellige komprimeringsalgoritmer, som for eksempel Huffman og PPM. En PPM komprimeringsalgoritme kræver en del mere af den enhed den kører på, i forhold til Huffman, og blev derfor ikke brugt til at udvikle løsningen, men i stedet, blev det bestemt, at den skulle udvikles med en Huffman komprimeringsalgoritme. Derudover blev det også bestemt, at Huffman skulle laves ud fra den statiske model. I forhold til både den dynamiske og den adaptive model, så krævede den statiske model langt mindre af den enhed, som den kørte på, så der derved ikke kom nogen problemer med, at enheden kørte dårligere på grund af programmet. 

Løsningen skulle også gerne være i stand til at køre under SMS-processen, sådan at brugeren af programmet ikke behøver at gøre noget ekstra, for at få SMS beskeden komprimeret eller dekomprimeret. Komprimeringsprocessen skulle gerne ske automatisk, idet brugeren afsender eller modtager en SMS besked. I løsningen er dette ikke tilfældet. Sådan som løsningen er lige nu, så spørger den efter hvorvidt, der skal komprimeres eller dekomprimeres. Løsningen kan derfor ikke køre under SMS-processen sådan at brugeren ikke ligger mærke til komprimeringen.