I forbindelse med projektet er der blevet tænkt over en række metoder, som gør os i stand til at finde ind til vores problems kerne, og understøtte problemet i sin helhed.
Ud fra projektforslaget er det blevet fastsat at vores endelige løsning til problemet skal være en prototype af et program. Denne prototype skal være skrevet i C, men da problemstillingen omhandler SMS-beskeder, er det  SMS-mediets krav, som vi skal designe vores løsning efter.


Ud fra vores problem findes der en række interessenter, som påvirkes af denne problemstilling.
På den følgende brainstorm, ses hvordan disse interessenter fordeler sig, ud fra det initierende problem, og hvordan de forbindes til hinanden.

IMAGE

Denne brainstorm identificerer vores primære interessanter, som vil være vores hovedmålgruppe.
Man kan dele interessenterne ind i nogle grupper, henholdsvis: Private personer, Internationale firmaer, Frivillige organisationer og Teleselskaber.
Hver af disse grupper har sin egen grund til at være interesseret i vores problemstilling, og derfor kan det også betyde, at der skal forskellige løsninger til at kunne løse problemstillingen, for hver forskellig interessant.


Ud fra vores problemstilling findes der en række data som kan være anvendelig i forhold til undersøgelsen af de førnævnte interessenter.


Viden om brugen af SMS’er hos de forskellige interessent grupper.
Det er vigtigt at finde ud af hvordan de forskellige interessenter bruger SMS’er som et medie. Med dette menes både hvor tit det bruges, men også i hvilken forbindelse og med hvem kommunikationen foregår.


Til indsamling af data omkring disse interessanter er det nødvendigt at komme i direkte kontakt med den målgruppe vi har med at gøre. Dette betyder at vi bliver nødt til at benytte nogle metoder, som gør det muligt at indsamle eller observere målgruppens forbrug af SMS’er.
Til dette vil en spørgeskema undersøgelse være velegnet, da brugen af SMS’er er data velegnet til kvantitative undersøgelser, da det er et spørgsmål om hvor mange SMS’er der sendes.
