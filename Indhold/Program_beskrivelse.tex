Vi har udviklet et program, der kan foretage komprimering, og dekomprimering af tekst, ved hjælp af Huffman komprimering. Teksten der bliver indlæst i programmet bliver komprimeret ud fra et statisk Huffman træ, genereret ud fra vores analyse af SMS-beskeder.


\subsubsection{Komprimering}

De data der skal komprimeres bliver læst fra en tekstfil på harddisken, ind i programmet. Disse data, bliver så sendt til komprimerings funktionen. Komprimerings funktionen, sørger så for at generere et Huffman træ, ud fra en fil på harddisken, med frekvenser for forskellige tegn. Dette kunne for eksempel være "gsmfreq.txt", lavet ud fra analysen af SMS-beskederne.
Når træet er lavet, bliver teksten så komprimeret, ved at den binære værdi for hvert tegn bliver skiftet ud med den tilsvarende binære Huffman sekvens. Denne sekvens, bliver sendt videre til en buffer, og hver gang denne buffer indeholder 8 bit (1 byte), skriver den til harddisken. Filen den den skriver til er en binær fil, og denne fil, vil efterfølgende kunne bruges til at genskabe de originale data (dekomprimering).

\subsubsection{Dekomprimering}

Dekomprimeringen af teksten ved hjælp af vores program, foregår ved nogle simple trin her beskrevet. Huffman træet, bliver igen genereret ud fra en fil med frekvenser. Denne fil skal være den samme, som da der blev komprimeret, da det genererede Huffman træ, skal være nøjagtigt magen til. Træet kunne i stedet have været gemt i en fil for sig selv.
Herefter påbegynder læsningen af den binære fil, der indeholder de komprimerede data. Der bliver læst bit for bit, fra venstre mod højre, indtil programmet ender på et tegn i Huffman træet. Dette tegn bliver skrevet til en tekstfil på harddisken, fx "output.txt". Algoritmen starter igen fra toppen af træet, og de næste bit i den binære fil bliver læst, og skrevet til disken, indtil alle data er dekomprimeret.

\subsubsection{Indeksering}
For at gøre komprimeringen hurtigere bliver træet indekseret. Dette gøres ved at køre én gang igennem alle kombinationer i træet, hvorefter placeringen af alle blade bliver tilknyttet dets GSM 7-bit værdi. Dette gør at når et tegn i inputstrømmen mødes, kan der lynhurtigt findes frem til dets blad, blot ved tegnets værdi. Dette er især nyttigt ved komprimering hvor der sidenhen kan komprimeres ved at køre opad i træet indtil roden mødes, mens der hver gang bliver gemt hvorvidt den aktuelle plads var forbundet til venstre eller højre side af den foregående. Når roden mødes har vi derfor den aktuelle Huffman kode baglæns, da der køres opad. denne kode spejles siden for at få den reelle kode, hvorefter den bliver skrevet til output strømmen.