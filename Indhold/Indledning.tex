I 1992 blev den første SMS afsendt \cite{museum}. Dengang kunne denne type tekstmeddelse maksimalt rumme 160 tegn, hvilket er nøjagtig lige så mange tegn, som en SMS kan indeholde i dag. Det var den tyske enginør Friedhelm Hildebrand, der i 1985 så potentialet i muligheden for at sende korte tekst beskeder fra mobiltelefon til mobiltelefon. Han undersøgte, hvor mange tegn der normalt blev brugt når man skrev et postkort, ligesom at han selv skrev forskellige beskeder, som han forestillede sig, at folk ville skrive til andre via deres mobiltelefon. I ingen af tilfældende var antallet af tegn i beskederne over 160. Derfor blev grænsen for hvor mange tegn en SMS kan indeholde sat ved 160.
Gennem tiden har meget teknologi ændret sig, så måske burde man i dag have muligheden for at sende flere end 160 tegn i en SMS. \cite{hillebrand}

\subsubsection {Datakomprimering}

Datakomprimering handler om at gøre en datamængde mindre. Der kan både være tale om filer, billeder, film osv.. Man ønsker ofte at data skal fylde så lidt som muligt, derfor er komprimering et meget centralt emne indenfor datalogi. Når man eksempelvis taler om hukommelseslager, computernetværk, herunder især internettet, er det meget relevant at data fylder så lidt som muligt.
 
Man kan komprimere enkelte filer såvel som hele samlinger af filer. Mange siger ofte at filer pakkes, når man komprimerer. Der findes allerede flere komprimeringsprogrammer; zip, jpeg, WinZip, SMS ZIP, SMS ZIPPER, for bare at nævne nogle få.

Der findes selvfølgelige mange måder at komprimere på og lige så mange forskellige komprimeringsalgoritmer. Disse algoritmer kan deles op i to kategorier, tabsfri og ikke-tabsfri.  
Det man ofte udnytter når man komprimerer tabsfrit, er at de fleste mængder af data indeholder den samme information flere gange. Eksempelvis indeholder en tekstfil de samme ord gentagne gange. Derfor kan man i stedet for den gentagne information, skrive informationer om hvor mange gange den bestemte information er blevet gentaget. På den måde kan dataene genskabes uden tab. Denne kategori benyttes især til tekstfiler, hvor det er særlig vigtigt at filen kan genskabes 100 procent. Ikke-tabsfri kompression kan eksempelvis benyttes til billedkomprimering, hvor det kan gå an at ikke hver pixel genskabes 100 procent.
