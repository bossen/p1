I 1992 blev den f�rste SMS afsendt \cite{museum}. Dengang kunne denne type tekstmeddelse maksimalt rumme 160 tegn, hvilket er n�jagtig lige s� mange tegn, som en SMS kan indeholde i dag. Det var den tyske engin�r Friedhelm Hildebrand, der i 1885 s� potentialet i muligheden for at sende korte tekst beskeder fra mobiltelefon til mobiltelefon. Han unders�gte hvor mange tegn der normalt blev brugt n�r man skrev et postkort, ligesom at han selv skrev forskellige beskeder som han forstillede sig at folk ville skrive til andre via deres mobiltelefon. Hverken antallet af tegn p� postkortene, eller antallet af tegn i de korte beskeder Hillebrand selv skrev overskreg 160. Derfor blev gr�nsen for hvor mange tegn en SMS kan indeholde sat ved 160. \cite{hillebrand}
Gennem tiden har meget teknologi �ndret sig, s� m�ske burde man i dag have muligheden for at sende flere end 160 tegn i en SMS.

\subsubsection {Datakomprimering}

Datakomprimering handler om at g�re en datam�ngde mindre. Der kan b�de v�re tale om filer, billeder, film osv.. Man �nsker ofte at data skal fylde s� lidt som muligt, derfor er komprimering et meget centralt emne indenfor datalogi. N�r man eksempelvis taler om hukommelses lager, computer netv�rk, herunder is�r internettet, er det meget relevant at data fylder s� lidt som muligt.
 
Man kan komprimere enkelte filer s�vel som hele samlinger af filer. Mange siger ofte at filer pakkes, n�r man komprimerer. Der findes allerede mange kompressions software; zip, gzip, WinZip, SMS ZIP, SMS ZIPPER, for bare at n�vne nogle f�.

Der findes selvf�lgelige mange m�der at komprimere p� og lige s� mange forskellige komprimeringsalgoritmer. Disse algoritmer kan deles op i to kategorier, tabsfri og ikke-tabsfri.  
Det man ofte udnytter n�r man komprimere tabsfrit, er at de fleste m�ngder af data indeholder den samme information flere gange. Eksempelvis indeholder en tekstfil de samme ord gentagne gange og et billede indeholder ofte den samme farve flere gange. Derfor kan man i stedet for den gentagne information skrive informationer om hvor mange gange den bestemte information er blevet gentaget.  P� den m�de kan dataene genskabes uden tab. Denne kategori benyttes is�r til tekstfil hvor det er s�rlig vigtigt at filen kan genskabes 100 procent. Ikke-tabsfri kompression kan eksempelvis benyttes til billedkomprimering, hvor det kan g� an at ikke hver pixel genskabes 100 procent.