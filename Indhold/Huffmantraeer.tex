Der er tre forskellige måder, eller træer, som man kan bruge Huffman coding til at komprimere et stykke tekst. De tre metoder generer enten et statisk, dynamisk eller adaptivt træ. Principielt ser det resulterende bit-træ ens ud for hver af metoderne. Det handler mere om hvordan træerne bliver generet. I dette afsnit vil der være en gennemgang af hvordan disse metoder virker, samt deres fordele og ulemper i forhold til SMS beskeder. Den følgende tabel samt det følgende billede viser hvordan et potentielt Huffman bit-træ kunne se ud.

%\noindent
\begin{table}[H]
\begin{center}
\begin{tabular}{|c|c|}
    \hline
    \cellcolor{ForestGreen}\color{white}{\textbf{Tegn}}\\[2ex] &  \cellcolor{ForestGreen}\color{white}{\textbf{Forekomster}}\\[2ex] \hline
    \ A & 24 \\hline
    \ B & 12 \\hline
    \ C & 10 \\hline
    \ D & 8 \\hline
    \ E & 8 \\hline
\end{tabular} 
\caption{Et sæt tegn og forekomster. Hentet fra binaryessence.com}
\end{center}
\end{table}

\begin{figure}[H]
\centering
\includegraphics[width=\linewidth]{Billeder/huffman_tree.png}
\caption{Det samme sæt som før set i et Huffman træ. Hentet fra binaryessence.com}
\label{fig:huffmantree}
\end{figure}

Ud fra billedet kan man se at de tegn som forekommer mindst er placeret nederst i træet, D og E, mens det tegn som forekommer mest er placeret øverst, A. Hver gang man går et niveau ned så går længden på et tegn op. A har en længde på 1 bit mens B, C, D og E har en længde på 3 bit. Tegnet som forekommer mest får altså den mindste længde. Derudover kan man også ud fra billedet se det følgende binære talsystem:

%\noindent
\begin{table}[H]
\begin{center}
\begin{tabular}{|c|c|c|}
    \hline
    \cellcolor{ForestGreen}\color{white}{\textbf{Tegn}}\\[2ex] &  \cellcolor{ForestGreen}\color{white}{\textbf{Binær Kode}}\\[2ex] 
    &  \cellcolor{ForestGreen}\color{white}{\textbf{Kode Længde}}\\[2ex] \hline
    \ A & 0 & 1 \\hline
    \ B & 100 & 3 \\hline
    \ C & 101 & 3 \\hline
    \ D & 110 & 3 \\hline
    \ E & 111 & 3 \\hline
\end{tabular} 
\caption{Binært talsystem fra et Huffman træ. Hentet fra binaryessence.com}
\end{center}
\end{table}

Dette sæt af tegn og forekomster fylder i alt 186 bit for sig selv. I et Huffman træ som dette billedet viser, fylder de samme tegn 138 bit som kan findes ved at lægge tegnene sammen i forhold til deres forekomster og længder. På billedet kan man også se punkter som ikke indeholder nogen tegn. Disse punkter kaldes for knuder og har en størrelse magen til summen af punkterne nedenunder den. Den øverste knude hvor træet altid starter kaldes for roden. Enderne på træet som indeholder de egentlige tegn kaldes for blade./cite{Hufftree_1}

\subsection{Statisk}
Den første metode man kan bruge til at generere et Huffman træ, er den statiske metode. Et statisk Huffman træ bliver lavet ud fra formodede forekomster af tegn i et stykke tekst. For eksempel hvis man kigger generelt på det engelske sprog, så forekommer tegnet ’e’ mest, og vil derfor blive placeret øverst i det statiske Huffman træ, mens tegn som ’z’ og ’x’ vil blive placeret nederst\cite{Hufftree_2}. Ligeledes kan man se i afsnit \ref{tegnForekomst} hvor tit tegn forekommer ud fra dette projekts undersøgelser. Den statisk opbygning af et Huffman træ er standard metoden.

Et statisk Huffman træ virker på alle stykker tekst, især længere stykker af tekst som for eksempel artikler. Til gengæld kan den ende med at ikke gøre fuld brug af komprimeringskraften ved Huffman coding, når det handler om mindre beskeder, fordi det ikke altid går op med normalen for tegn forekomster i et sprog. Statiske Huffman træer virker bedre, i takt med at den tekst som skal komprimeres bliver større, og bliver mere ineffektiv, når teksten bliver mindre. Det er ikke sandsynligt, at den komprimerede tekst vil fylde mere, end hvis den ikke var komprimeret idet at Huffman komprimering er så effektivt som det er. \cite{Hufftree_3}

Fordele ved et statisk træ er at det let kan give gode resultater for større stykker af tekst og behøver ikke at sende noget yderligt i beskeden udover bitmønstret. Et statisk træ er også hurtigt idet at der ikke er behov for at generer et nyt træ for hvert eneste stykke tekst, som der er behov for med de to kommende metoder. Statisk kan derfor have en fordel, fordi den ikke behøver ligeså meget regnekraft på enheder som mobiltelefoner, som muligvis kan være begrænset på det område.


\subsection{Dynamisk}
Den anden metode man kan bruge kaldes for et dynamisk Huffman træ. Et dynamisk træ bliver genereret ud fra den reelle data, som teksten består af. Træet bliver genereret ud fra den specifikke besked, og er ikke en generel liste som ved den statiske metode. Det dynamiske træ er derfor det mest optimale træ til Huffman coding. Den dynamiske opbygning af et træ starter med at sætte de to tegn med færrest forekomster sammen til en knude. Derefter bliver den knude sat sammen med det næste tegn, som forekommer mindst og er endnu ikke sat ind i træet, og sat til en ny knude. Denne process fortsættes indtil alle tegn er blevet sat ind i træet, og der er en rod i toppen af træet. Figur \ref{fig:dynamic_tree} viser dette.

\begin{figure}[H]
\centering
\includegraphics[width=\linewidth]{Billeder/dynamisk.png}
\caption{Dynamisk opbygning af et Huffman træ udfra det tidligere eksempel \cite{Hufftree_1}}
\label{fig:dynamic_tree}
\end{figure}

En ulempe ved den dynamiske metode er, at for at kunne dekode den komprimeret tekst, skal man bruge en kode tabel over træet. Idet at det er et træ skræddersyet til et bestemt stykke tekst, så har dem der skal dekode teksten ingen mulighed for at oversætte 0 og 1 tallene tilbage til den oprindelige tekst. Derfor er det nødvendigt, med den dynamiske metode, at medsende en tabel som angiver hvilke tegn der har hvilke bitmønstre. Dette betyder selvfølgelig, at den komprimerede tekst fylder et stykke mere, og gør derfor ikke optimalt brug af komprimeringskraften ved Huffman coding. \cite{Hufftree_4}

Fordelen ved et dynamisk træ i forhold til det statiske er, at den mere konstant gør optimal brug af Huffman coding til at komprimere et stykke tekst. Det er selvfølgelig det man gerne vil opnå i forhold til SMS-beskeder, men som sagt så er det nødvendigt at sende en tabel med, så det er muligt at dekode den komprimerede tekst. Dette betyder, at SMS-beskeden fylder mere end hvis det bare var det komprimerede stykke tekst, og arbejder derfor imod målet om at formindske størrelsen, af den data der bliver sendt.

\subsection{Adaptivt}
Den tredje og sidste metode til at komprimere et stykke tekst med Huffman coding er den adaptive metode. Med den adaptive metode starter træet med at intialisere forekomsten af tegn som de kommer til 1. Når det samme tegn fremkommer igen så vil det blive lagt til tegnets fremkomst værdi, og træet vil derefter også opdatere sig selv sådan hvis et nyt tegn begynder at have flest forekomster så vil den blive placeret øverst i træet.
I forhold til den foregående metode så kan den adaptive metode genere et træ alt efter hvad for noget data som skal komprimeres, ligesom den dynamiske metode, men har til gengæld ikke behov for at videregive en tabel som viser hvordan man oversætter det komprimerede tekst, idet at det adaptive træ på den enhed som dekomprimere vil gøre arbejdet og opdatere sig selv med det data den dekomprimere. Dette kan også betyde at det dynamiske træ muligvis ikke behøver at sende en tabel med over hvis dekomprimeringen sker ved hjælp af det adaptive træ. Ulempen ved den adaptive metode er at det stykke tekst der bliver komprimeret eller dekomprimeret ikke indgår i den nuværende version af træet, idet at det adaptive træ først opdateres med data fra tekststykket efter det er blevet komprimeret eller dekomprimeret. Derudover så kræver den også meget regnekraft fra den enhed som skal komprimere og dekomprimere fordi den konstant skal opdateres af det data den gennemarbejder. /cite{Hufftree_5}
I forhold til SMS beskeder så er den adaptive metoder sikkert bedst når det kommer til at formindske den mængde data der bliver sendt med beskeden. Ulempen er dog at træet har brug for tid til at komme op og køre ved at konstant opdatering af sit eget træ. Den adaptive metode starter derfor ud med at være langsom og ineffektivt idet at den introducere alle tegn fra bunden af og kan først effektivt komprimere efter den har arbejdet sig igennem et antal tekststykker. Kravet for regnekraft på enhederne der komprimere og dekomprimere  kan også være en større ulempe på en mobiltelefon.