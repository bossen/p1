Som nævnt i indledningen er der to former for komprimering, tabsfri og ikke-tabsfri metoder. 

Tabsfri er, som navnet antyder, en metode hvor ingen data går tabt i komprimeringsprocessen. Den komprimerede og senere dekomprimerede data er altså eksakt magen til den oprindelige data. Alle tabsfri komprimeringsmetoder bryder filen op i mindre sektioner, og udnytter redundans. Redundans er fx når ord eller bogstaver optræder oftere end nødvendigt. Alt dette bevirker at via en algoritme kan dataen genskabes perfekt ved dekomprimeringen. En af ulemperne ved denne komprimeringsmetode er at kompressionsratioen er lavere end en ikke-tabsfri metoder\cite{wisegeek}. De tabsfri metoder bruges, som i vores tilfælde, til kompression af tekst. Ydermere bruges de til billeder af for eksempel formatet PNG. 

Ikke-tabsfri komprimering genskaber derimod ikke en eksakt kopi af den oprindelige fil. Ved ikke-tabsfri komprimering fjernes unødvendig data, fx data der opstår flere gange i filen. Fordelen ved disse metoder er ulempen ved de tabsfri metoder, altså at kompressionsratioen er højere end de tabsfri. Ikke-tabsfri metoder bruges som oftest til lydfiler, billed- og videofiler\cite{maximum}. Billedfilerne er naturligvis af andre formater end dem ved tabsfri komprimering, det kunne være filer som fx JPEG.
