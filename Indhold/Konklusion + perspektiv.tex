Dette projekt omhandler tekstkomprimering og hvordan man kan bruge dette i forbindelse med SMS-beskeder, som har en øvre grænse for hvor mange tegn man kan sende pr. besked. Den første del af vores problemformulering lyder;
\emph{Hvordan kan man spare forbrugeren for dobbelt SMS-takst, ved brug af et komprimeringsprogram?}
Til at besvarer dette spørgsmål, hvar vi udviklet et program, som kan komprimere en tekstbesked ved brug af Huffman-algoritmen. Som det ses i afsnittet RESULTATER er det lykkedes af komprimere bedskeder med over 50 procent, hvilket betyder at det, med brug af programmet, er muligt at sende to SMS-beskeder som én. På den måde vil brugeren kunne spare SMS-takst, hver gang der skal sendes en SMS på over 160 tegn. 
Næste del af vores problemformulering lyder;
\emph{Hvordan kan man undgå at programmet bliver en belastning for brugeren?} og dette 

Perspektivering
Løsningen på problemet er som beskevet et komprimeringsprogram. Programmet komprimerer tekstbeskeder, så de fylder mindre og man på den måde kan sende flere tegn i den samme SMS. Når en SMS afsendes, sendes denne som en komprimeret tekststreng og som sådan vil den også modtages af modtageren af beskeden. Derfor kræver vores løsning at modtageren også har programmet. Hvis man som bruger sender komprimerede beskeder skal man være sikker på at motagerene af SMS-beskederne ligeledes har programmet, ellers giver beskeden ingen mening for dem. En løsning på dette problem, kunne være at lave en opdatering, med programmet, således at alle kunne opdatere deres mobiltelefon og derved have programmet. 