Til dette projekt skal der udarbejdes en l�sning i form at et program, som kan komprimere en kort tekstbesked. Komprimeringen vil g�re tekstbeskeden mindre, det vil sige beskeden fylder f�rre bytes, og skulle gerne b�de g�re det hurtigere at sende beskeden fordi den er mindre, men ogs� g�re det muligt at sende en besked over en bestemt tegn begr�nsning, som f. eks. de begr�nsede 160 tegn ved brug af det latinske alfabet i en SMS. Beskeden skal derefter dekomprimeres hos modtageren, og derefter vise beskeden, som den s� ud f�r den blev komprimeret. Denne process skal ske uden brugeren selv tager en direkte del i processen.

\begin {itemize}
\item Funktionelle Krav
\subitem Skal b�de v�re i stand til at komprimere og dekomprimere automatisk.
\subitem Programmet skal v�re i stand til at skelne mellem hvorvidt den p�g�ldende besked skal komprimeres eller dekomprimeres.
\subitem Det er ikke forventet at prototypen skal kunne k�re p� en mobil enhed som f. eks. en smartphone, men det er forventet at programmet kan bruges p� en computer.

\item Ikke Funktielle Krav
\subitem Produktet skal afleveres sammen med den tilh�rende rapport, og har en f�lles deadline den 19 December 2012.
\subitem Programmet skal skrives i programmeringssproget C.

\item L�sningsm�l
\subitem Brugeren skal kunne g�re brug af programmet uden selv at tage direkte del i komprimerings processen.
\subitem Programmet skal k�re lokalt, og ligeledes skal komprimeringen og dekomprimeringen skal ogs� ske lokalt.
\subitem Programmet skal implementeres og v�re brugbart.
\end{itemize}