Entropikodning er en lossless/tabsfri datakomprimeringsmetode. Tabsfri, betyder at der ikke g�r nogen information tabt, ved at komprimere datam�ngden. Modsat har vi lossy/tabsgivende komprimering, som fx MP3, og JPEG. Entropopikodning g�r ud p�, at f� en given datam�nde til at benytte et minimalt antal bit. \cite{entro1} Dette kan opn�s ved at kigge p� hyppigheden af de forskellige tegn i datam�ngden der skal komprimeres, og give de oftest fremkommende symboler f� bits, og de mere sj�ldne symboler flere bits. Form�let er, at f� det gennemsnitlige antal bits pr. symbol(middelkodel�ngden) ned. Den teoretiske nedre gr�nse for middelkodel�ngden kaldes datam�ngdens entropi.
Der findes flere forskellige entropikodningsmetoder, og et par eksempler er "Huffman kodning" og "arithmetic coding".