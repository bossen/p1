\documentclass[12pt]{report}
\usepackage{a4}
\usepackage[T1]{fontenc}
\usepackage[danish]{babel}
\usepackage[hidelinks]{hyperref}
\usepackage{lingmacros}
\usepackage{tree-dvips}
\usepackage{url}
\usepackage{graphicx}
\usepackage{float}


\begin{document}

\chapter*{Gruppekontrakt}
\section*{Ambitionsniveau}
Der er fastsat h�jt ambitionsniveau i gruppen.
\section*{M�dedisciplin}
Der er m�depligt ved alle m�der, b�de vejlederm�der og gruppem�der.
Man har pligt til at melde afbud, hvis man ikke kan komme. Dette g�lder ogs� hvis man ikke kommer til en forel�sning.
\section*{Fredagshygge}
Der er aftalt at der hver fredag medbringes kage, for at �ge motivationen. Gruppemedlemmer har skiftevis ansvar for at medbringe kage, hvilket foreg�r i f�lgende r�kkef�lge:
\begin{enumerate}
	\itemsep1pt \parskip0pt \parsep0pt
	\item Anders
	\item Rikke
	\item �lavur
	\item Frederik
	\item Jannek
	\item Katrine
	\item Kevin
\end{enumerate}
\section*{Tidsplaner og Deadlines}
Projektets ultimative deadline ligger p� onsdag d. 19. december. Der vil dog fremg� del-deadlines i tidsplanen. Dette g�r endvidere at alle i gruppen er afklaret med hvad der skal v�re f�rdigt til enhver tid.
\section*{Problemer og Konflikter}
Det er op til den enkelte gruppemedlem at f� den p�g�ldendes arbejde f�rdigt til tiden. Det er derfor op til gruppemedlemmen at sige til hvis han/hun ikke er helt afklaret om arbejdsopgaven. Der er dog mulighed at samarbejde om opgaver hvis det er n�dv�ndigt. Det er endvidere op til den enkelte gruppemedlem at give status p� en arbejdsopgave ved gruppem�der.
\section*{M�destruktur}
Der vil ved m�der v�re en dagsorden, der er fastlagt p� forh�nd. Det er dog muligt at lave tilf�jelser til dagsordenen under m�det om n�dv�ndigt. Under vejlederm�der vil der blive udvalgt en referant der tager noter under m�det. Efter m�der vil der blive reflekteret m�det og det videre arbejde vil blive dr�ftet samt uddeling af arbejdsopgaver.
\section*{Socialt samv�r}
Hvad skal st� her?
\end{document}